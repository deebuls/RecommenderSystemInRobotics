\section{Introduction} 
\acrfull{lfd} is a technique of learning new skills based on
demonstrations by a teacher\cite{argall_survey_2009}. In 2013, robot sales
increased by 12 \% to 178,132 units, by far the highest level ever recorded for
one year. Sales of industrial robots to the automotive, the chemical, and the
rubber and plastics industries, as well as to the food industry continued to
increase in 2013. Between 2008 and 2013 the average robot sales increase was at
9.5\% compounded annual growth rate per year
\footnote{\url{http://www.ifr.org/uploads/media/Executive_Summary_WR_2014_02.pdf
}}. With the increasing number of robots being deployed in the industries the
amount of efforts to program the robots is also increasing. It is a major
requirement to program the robots with new tasks even by persons who are non
experts in robotics. \acrshort{lfd} comes as an appropriate solution to this problem, in
which a non expert can demonstrate the expected skill to the robot and the
robot can learn from it.


In our work we work on the problem of inferring the relevant aspects of the
demonstration for a successful reproduction of a motion primitive. A motion
primitive is defined as a single elementary movement
\cite{schaal_nonlinear_2000}. Motion primitive forms the building blocks for
the robotic skills, which in turn forms building blocks of robotic tasks. For
example we want the robot to do a task \textit{"Get milk bottle from the
refrigerator."}. This task can be broken down into various robotic skills
\textit{moveto(refrigerator) + open(refrigerator) + pickup(milk bottle) +
close(refrigerator) + moveto(table)}. Each of the robotic skills can be
composed of several atomic movements i.e. motion primitives . So the skill of
\textit{pickup} can be composed of \textit{$move\_arm$(milk bottle) + grasp(milk
bottle) + $move\_arm$(tray) + release(milk bottle)} motion primitives. Thus each
task of the robot can be broken down to the level of atomic actions of the
motion primitives. The motion primitives are executed in a sequence to achieve
the higher goals.

Motion primitives in robotics is a highly researched topic. There are various
representation of motion primitives available in the literature. In our
approach we propose a feature based representation for the motion primitive. A
feature is a quantitative element of the world the robot can measure. In our
approach we identify the relevant features that completely describe a motion
primitive. The general approach taken in machine learning for the problem is to
create a large training dataset and use feature selection algorithms to
determine the relevant features. But this approach cannot be used in \acrshort{lfd} since
the demonstrations which are the training dataset, are sparse.


In \cite{abdo_inferring_2014} came up with a novel idea for determining
relevant features using few demonstrations by taking inspirations from
recommender systems . They build an off-line expert knowledge base using
experts, which is similar to the user preferences created by recommender
system. Now the expert knowledge base is used to make a recommendations on the
relevant features of the motion primitive.


Our work advances the approach of \cite{abdo_inferring_2014}, such that the
knowledge base created is more structured using effect metrics. Effect metrics
are a means of measuring changes which have occurred in the robot, environment
and interactions of robot-environment due to an action.  The knowledge base is
collected off-line and before any demonstrations. When a new motion primitive
is learned, the demonstrations are shown by an user. The robot, based on the
data available from demonstrations and the off-line knowledge base makes
recommendations on what are the relevant features for the motion primitive.


\subsection{Why not the term "recommender system" ?}
The work should be ideally named as "Learning by Demonstration, recommendation
using Expert knowledge base". The name "recommender system" is an established
term in literature. Recommender systems are a subclass of information filtering
systems that seek to predict the 'rating' or 'preference' that a user would
give to an item \cite{bobadilla_recommender_2013}. The prediction of the
preference is done based on a knowledge base generated using the user data.
This knowledge base generation is always an online process. But in our work we
try to create the knowledge base off-line that is based on expert knowledge.
Also in our case the system recommends set of features that could be used to
explain a demonstrated motion primitive by doing a greedy search on all the
templates available in the knowledge base. This is not the case in recommender
systems. So even though in principle the idea of both the systems are same, we
would like to refrain from using the term recommender system and rather use a
more generic term "recommendations using knowledge base".
