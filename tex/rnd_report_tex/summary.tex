\section{Summary} 
In this work we presented relevant feature selection based on
a knowledge base recommendation. The proposed approach was an 
extension to the framework proposed by \cite{abdo_inferring_2014}.
We also have proposed a possible framework for representing a motion primitive 
in a skill based framework by \cite{pedersen_robot_2015} . 
The method was evaluated on experiments conducted on simulation and 
the real robot

\begin{itemize}
    \item \textbf{Successful learning on fewer demonstrations} \\
       The result showed that we could successfully learn on fewer
       demonstrations. All the experiments conducted the relevant
       features could be recommended using just \textbf{three} demonstrations.

    \item \textbf{Structured nature of the knowledge base helps in recommending} \\
        The knowledge base was re structured to segregate templates based on the 
        effect metrics. This ensured that the correct templates are made available 
        for recommendations. This structure of the knowledge base acted as a  catalyst 
        for recommending in less number of demonstrations.

    \item \textbf{Generalization to un-demonstrated start states}\\ Our results
        show that robot trained using our recommendations not only learns the
        goal on demonstrated motion primitives, but also generalizes well to do
        motion primitives from initial configurations not demonstrated before.

    \item \textbf{Use of proposed method to non-experts}\\
        Using the knowledge base generator \ref{knowledge base generator}
        we could easily add new features and templates. This helps
        in ease for non-experts to teach the robot new motion primitives.

    \item \textbf{Scalability of the approach}\\
        The approach was specifically made to work under the skill based framework
        \ref{apx:skill based framework} , the approach is highly scalable.
        New motion primitives can be easily added and the existing motion
        primitives can be resued to form new skills. Specially in the 
        case of industrial robotics, the author claims that it can be 
        utilized for solving majority of the tasks.

\end{itemize}


Our application scenarios of learning goal from demonstrations goes beyond the
application scenarios that previous work has considered.

\newpage
\section{Limitations of work}

This section lists some of the major limitations of the approach.
\begin{itemize}
    \item \textit{Poor quality of demonstration} \\
As \acrshort{lfd} systems are totally dependent on the demonstration of the data, its 
quality greatly reflects on the results. Poor quality of demonstration will lead to 
very poor results. For example in our approach its very much necessary that there has 
to be significant ddifference in the features which are not to be learned. So the 
demonstrations has to vary for all the features not being learned. So great care has to 
be taken by the user to teach a correct set of demonstration.

    \item \textit{Lack of element for feedback} \\
The proposed approach there is not provisions for providing feedback to the robot after
the learning. This is major limitation in terms of learning. Feedback are a good way for 
the robot to generalize on new situations.

    \item \textit{Lack of continuous learning} \\
The proposed approach the learning is done in a single shot. There is no provision for 
feedback and continuos learning. A scalable approach would be if there are provisions 
for the robot to continuos learn while executing. Such provisions will ensure that the 
robot is continuously learning and improving its decision making.


\newpage

\end{itemize}
\section{Future Work}
This section proposes areas that can be improved or extended for future works 
in the field

\begin{itemize}
    \item \textit{Learning "When to imitate" } \\
        The current work only focussed on the "what to imitate" part for identifying
the relevant aspects of the demonstrations. "When to imitate" is a section of 
\acrshort{lfd} to determine when the demonstrated action should be executed.
The proposed approach can be used to identifying the start conditions (pre-condition)
by identifying the relevant aspects in the start values of the demonstrations.

    \item \textit{Better evaluation strategies} \\
        The current evaluation method is manual and are subject ot faults. Automated
methods for evaluting the results need to be used for better confidence on the results.

    \item \textit{Combining both entropy and conditional entropy } \\
The current results showed that entorpy based relevance performed better while
conditional entorpy didnt perform better for all the cases. But conditional entropy 
performed better  in some cases. This needs to be further investigated why the 
conditional entropy didnt perform better in some experiments and how we can modify 
the approach to use it for getting better results.

    \item \textit{Creating a complete skill based on the proposed motion primitive framework}\\
A motion primitive framework is proposed in this work. The relevant features were also 
learnt. Now the work has to be extended to use the relevant feature and execute a particular
motion primitive successfully. Also a set of motion primitives has to be learnt and 
a complete skill has to be executed.

    \item \textit{Better \acrfull{hmi}} \\
The current methods of recording the demonstrations are not sutiable for outside lab.
Since one of the big benefits of \acrshort{lfd} is that non-experts would be able to 
teach the robot with new skills, its of high importance to work on better more intuitive 
\acrshort{hmi} needs to be developed.

    \item \textit{Learning of the templates on-line} \\
The templates used in the approach is generated manually using experts of manipulation.
But in future these futures has to be learnt on-line. The base templates can be used 
from the off-line process but these needs to be updated online based on the demonstrations.
For example for a motion primitive representing the distance between tool-tip and object template will 
be recommended, also if joint 4 is showing relevance as in all the demonstrations joint 4 
was show in a particular angle, then it should be added on-line to the 
template list.

    \item \textit{Completenes of the approach} \\
The results of the approach shows positive signs of the usefullness of the approach.
But we need to have a critical look at the completeness of the solution i.e. can the
approach be generalized for learning all the motion primitives required for a general 
robot. Works like \cite{bogh_does_2012} have claimed that for all the task in an 
industrial setup, a limited list of skills is sufficient. The completeness of our 
approach needs to be verified.

\end{itemize}



